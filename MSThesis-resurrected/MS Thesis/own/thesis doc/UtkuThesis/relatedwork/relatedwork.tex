% RELATED WORK
\chapter{Related Work}
\label{chapter:relatedwork}

The GRN control problem is a popular problem that has gained more attention with the advancements in
molecular biology in the last decade. The idea of intervening with the genes became possible with the
research on TFs and other dynamics in the protein biosynthesis. From the biological point of view, simple
pathways of genes were formulated for explaining certain phenomena in the~cell.

Lately, the research community has focused more on generic and complex interactions between genes, and GRNs
became an important tool; intervention problems have been formulated using GRNs~\cite{Datta03,Pal08,Pal06}.
Different mathematical frameworks and approaches have been used for devising mechanisms of intervention with
genes in order to accomplish some non-trivial goals. There are numerous useful examples of this intervention
mechanisms. For instance, targeted therapy is a cancer treatment technique based on suppressing the
expression of some genes in order to prevent cancerous behavior in cells~\cite{Green04}. Employing
mathematical models and computational techniques in undertaking the intervention problem would increase our
ability to devise more complex intervention~mechanisms.

Modeling the GRN control problem using PBNs is an approach widely used by the research
community~\cite{Kauffman93}. PBN model network dynamics and different approaches can be used for solving
control problems defined on this representation. Expressing the problem in terms of an MDP and using general
purpose MDP solving algorithms is an approach well suited to PBN representation. Examples of methods that use
Markovian and MDP concepts are the work of Shmulevich et al.~\cite{Shmulevich02} and the work of Datta et
al.~\cite{Datta03} who studied the dynamics of PBN model in the probabilistic context of Markov chains. The
work of Pal et al.~\cite{Pal05} explores an alternative model, namely context-sensitive probabilistic Boolean
network for the intervention~problem.

Our research group has already contributed to the control and intervention of GRN by developing novel
approaches based on PBN formulation of the network and different formulations of the intervention problem,
including MDP formulation~\cite{Abul04,Abul06,Tan08,Tan10,Tan10b}. The main focus on these previous works by
our group was formulating the GRN in PBN model and trying to solve the MDP problem in different settings and
exploring different aspects such as finite or infinite horizon reward mechanisms, factored representations of
the MDP problems, and improved modeling and solution techniques for plain and factored MDP problems.
Although our techniques have been effective and well received by the research
community~\cite{Abul04,Abul06,Tan08,Tan10,Tan10b}, we realized the need for incorporating partial
observability in the problem definition; and accordingly, the target of our approach described in this paper
is to develop appropriate solutions for the problem augmented to be partially~observable.

The need to cover partial observability has been realized by some other researchers; however, the problem has
not yet received enough and comprehensive attention. Datta et al.~\cite{Datta03} developed a finite horizon
control algorithms for GRNs; their algorithms can work with imperfect information. They modeled the gene
regulatory network as a PBN and used Markovian probabilities for their control algorithm, which is basically
an optimization algorithm based on dynamic programming. Bryce et al.~\cite{Bryce07} were first to model the
GRN control problem as a POMDP. Their approach makes use of PBN representation of the problem and their POMDP
formulation is similar to the PBN formulation. Their focus was on solving finite horizon problems, and
obviously they did not use a POMDP solver for solving the derived POMDP problem. They applied a probabilistic
planning algorithm for extracting an intervention mechanism for a finite horizon. Our approach described in
this paper is a major contribution to the literature because to the best of our knowledge it is the first
effort to develop a method that properly and comprehensively incorporates partial observability in handling
the control~of~GRNs.
